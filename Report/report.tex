\documentclass{article}

\begin{document}
\title{Porting the Jacobi Code to OpenMP}
\author{Adam Matheson \\
Student Number: P 1757290}
\date{\today}
\maketitle

\section{Introduction}
The trend in modern HPC is clear: parallelization is the way
forward. In every aspect, performance is being found through
concurrent execution; be that via vectorisation, multi-core CPUs or
GPGPU computing. Despite new challenges arising from this new
paradigm, the benefits on offer make this new way of programming more
than worth the effort.

To illustrate the potential performance gains on offer, what follows
is a walkthrough of a successful attempt to parallelize optimised
serial code running the Jacobi algorithm for solving a set of linear
equations, by porting it to OpenMP.

\subsection{Methodology} After making a change, the program was
re-profiled and the new run-time for matrices of orders 500, 1000,
2000 and 4000 recorded, averaged and rounded to an appropriate
accuracy across three runs on BlueCrystal. Times are given in the
format W/X/Y/Z for each matrix order respectively and are in
seconds. For convenience, where significant performance gains were
made, improvements are described in terms of ``\(t\)X times faster''
(where \(t\) is the speedup); but smaller improvements are described
in terms of percentages.

\subsection{Serial Optimisations}
Prior to the parallelization described in the following sections, a
number of basic changes and serial optimisations were
explored. Specifically, these were (in order, performance gain in
reference to times after previous optimisation):

\begin{enumerate}
\item Change compiler from GNU (gcc 4.8.5) to Intel (icc 18.0.0): ~3X
  faster
\item Ensure data order and data access pattern are the same (column
  vs. row traversal): ~3.5X faster
\item Change data type from \texttt{double} to \texttt{float}:
  ~5\%-10\% faster
\item Enable \texttt{-O3} compiler optimisation flag: ~15\% faster
\item Vectorisation check using \texttt{-qopt-report}
\item Attempted loop fusion (failed: introduced high error rate)
\end{enumerate}

\subsubsection{Further Serial Optimisations}
In addition to the above, further potential optimisations became
apparent after submission and were dutifully included in the serial
code before embarking on this parallelization endeavour:

\begin{enumerate}
\item Move branching code \texttt{if (row != col)} in \texttt{run()}
  method to outer loop
\end{enumerate}

These originally non-included optimisations resulted in a reduction of the runtime from 0.148/1.107/7.879/72.155 when only using my original submission's optimisations, to W/X/Y/Z.

\section{Parallelisation}

\subsection{Baseline}
To judge relative performance improvements, we require a baseline
runtime. The serial code with the above optimisations and changes
(including continuing with \texttt{icc} and \texttt{-O3} optimisation)
ran in W/X/Y/Z.

\subsection{Performance Analysis}
Relative performance improvements are good, but how does a programmer
know if he is really running fast code? What is ``fast''? We require a
way to measure absolute performance gains, against a theoretical
maximum performance. We can use the concept of ``operational
intensity'' to produce a few approximations of a theoretical maximum
performance.

\subsubsection{Computational Complexity}
It is useful to first understand the computational complexity (CC) of
the Jacobi code in terms of ``big O'' notation. The key part of the
Jacobi algorithm processes a matrix, nestling a \ldots therefore the
CC is \ldots

\subsubsection{Operational Intensity}
We can use the CC to help work out the operational intensity
(OI). This is a measure that can be described as ``operations per byte
of memory traffic''. Whereas CC only accounts for compute cost, OI can
also describe the relationship between compute and memory cost.

Looking at the Jacobi code and its CC, we can also identify the memory
operations required during matrix processing. Combining this with the
CC gives OI of \ldots

\subsubsection{STREAM Benchmark}
Another way of measuring absolute performance is with the STREAM
benchmark. This measures the sustainable memory bandwidth for four
long vector operations on specific hardware. Knowing the result of
this benchmark for BlueCrystal can tell us how close we are to
achieving peak memory bandwidth.

Running STREAM on BlueCrystal results in \ldots

\subsubsection{Roofline Model}
Now the OI and peak memory bandwidth can be used in a roofline
model. This graph provides a visual means of identifying optimal
performance in terms of a trade-off between compute and memory
cost. We can also identify where the current code is under the
``roof'' of the graph to determine whether the code is compute-bound
or memory-bound, helping to focus optimisation and parallelization
efforts. Here is the roofline model for Jacobi running on BlueCrystal:

\ldots

\subsection{Adding Pragmas}
parallel for loop
look at toy code

\subsection{Profiling}
gperftools, tau

\subsection{Cache coherency/false sharing}
Reduction

\subsection{Reprofiling}
Overhead of parallel

\subsection{Libraries, BLAS, NAG C}
Seven dwarves

\subsection{Re-testing compiler}
Sometimes compiler optimisations can be arbitrary, worth checking flags again

\section{Conclusion}
\subsection{Going Further - Scaling}
One final useful thing to know is how well a piece of code will scale
with further parallelization. Testing the Jacobi algorithm on a growing number of
cores results in the following graph:

\ldots

Which means the scaling is type\ldots

So can be scaled more/cannot be scaled that well

Amdahl's Law Gustavson's Law

\end{document} 